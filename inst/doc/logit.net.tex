\documentclass{article}

\title{
  logit.net: Network Logistic Regression for Dichotomous Proximity Matrix 
  Dependent Variables
}

\author{Matt Owen, Olivia Lau, Kosuke Imai, and Gary King}



\usepackage{bibentry}
\usepackage{graphicx}
\usepackage{natbib}
\usepackage{amsmath}
\usepackage{url}
\usepackage{Zelig}
\usepackage{Sweave}

%\VignetteIndexEntry{Network Logistic Regression for Dichotomous Proximity Matrix Dependent Variables}
%\VignetteDepends{Zelig, stats}
%\VignetteKeyWords{model,least squares,continuous, regression}
%\VignettePackage{Zelig}

\begin{document}
\nobibliography*


\section{{\tt logit.net}: Network Logistic Regression for Dichotomous Proximity Matrix Dependent Variables}

Use network logistic regression analysis for a dependent variable that is a binary valued proximity matrix (a.k.a. sociomatricies, adjacency matrices, or matrix representations of directed graphs). 

\subsubsection{Syntax}
\begin{verbatim}
> z.out <- zelig(y ~ x1 + x2, model = "logit.net", data = mydata) 
> x.out <- setx(z.out)
> s.out <- sim(z.out, x = x.out)
\end{verbatim}



\subsubsection{Examples}
\begin{enumerate}
\item Basic Example

Load the sample data (see {\tt ?friendship} for details on the structure of the network dataframe):

\begin{Schunk}
\begin{Sinput}
> data(friendship)
> 
> 
> 
\end{Sinput}
\end{Schunk}
Estimate model:

\begin{Schunk}
\begin{Sinput}
> z.out <- zelig(friends ~ advice + prestige + perpower, model = "logit.net", data = friendship)
> summary(z.out)
> 
\end{Sinput}
\end{Schunk}
Setting values for the explanatory variables to their default values:

\begin{Schunk}
\begin{Sinput}
> x.out <- setx(z.out)
> 
\end{Sinput}
\end{Schunk}
Simulating quantities of interest from the posterior distribution.
\begin{Schunk}
\begin{Sinput}
> s.out <- sim(z.out, x = x.out) 
> summary(s.out) 
> plot(s.out) 